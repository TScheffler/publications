\documentclass[citeauthoryear]{llncs}

%%%%%%%%%%%%%%%%%%%%%%%%%%%%%%%%%%%%%%%%%%%%%%%%%%%%%%%%%%%%%%%%%%
\usepackage{makeidx}  % allows for indexgeneration
\usepackage{paralist} % for inparaenum
\usepackage{multirow} % for multirow
\usepackage{tabularx} % for centering in table
\usepackage{url}
%%%%%%%%%%%%%%%%%%%%%%%%%%%%%%%%%%%%%%%%%%%%%%%%%%%%%%%%%%%%%%%%%%
\newcommand{\totaloov}{\% of total OOVs}
\newcommand{\uniqoov}{\% of unique OOVs}
%%%%%%%%%%%%%%%%%%%%%%%%%%%%%%%%%%%%%%%%%%%%%%%%%%%%%%%%%%%%%%%%%%
\begin{document}
\pagestyle{headings}  % switches on printing of running heads
\title{MT @HanBaldwin: Fightin OOVs in German \#twitter}
\titlerunning{Makn Sens a Germa \#twitter}  % abbreviated title (for running head)
%                                     also used for the TOC unless
%                                     \toctitle is used
%
\author{Uladzimir Sidarenka \and Tatjana Schef\/f\/ler \and Manfred Stede}
%
\authorrunning{Sidarenka et al.} % abbreviated author list (for running head)
%
%%%% list of authors for the TOC (use if author list has to be modified)
\tocauthor{Uladzimir Sidarenka, Tatjana Schef\/f\/ler, Manfred Stede}
%
\institute{University of Potsdam,\\
  \email{{uladzimir.sidarenka@uni-potsdam.de,
      tatjana.scheffler@uni-potsdam.de,
      manfred.stede@uni-potsdam.de},\\ WWW home page:
    \texttt{http://www.ling.uni-potsdam.de/acl-lab/SocMedia/main.htm}}}

\maketitle              % typeset the title of the contribution

\begin{abstract}
  %% Since its launch in March 2006 and until today Twitter constantly
  %% gained more and more popularity as a communication means among the
  %% Internet users. As of the end of 2012, approximately half a billion
  %% messages were published every day by using its services (Terdiman,
  %% \cite{terdiman}).  But, unfortunately, this abundant resource of
  %% textual information can not be exploited to the full extent by
  %% solely relying on standard NLP frameworks, because misspellings,
  %% active usage of slang, and various other aspects of texting language
  %% significantly decrease the chances of successful automatic analysis
  %% of this kind of data.

  This article gives an overview of existing approaches to the problem
  of out-of-vocabulary (OOV) tokens and noisiness phenomena in natural
  language texts. These approaches are classified with regard to the
  size of text spans and knowledge inference mechanisms which they use
  in their work. Subsequently, quantitative and qualitative analyses
  of unknown words are conducted on German Twitter messages, in order
  to see how relevant the OOV issue is for this particular kind of
  micro-texts and to determine which peculiarities are especially
  characteristic for them. In a concluding step, we present a set of
  ad-hoc techniques which are supposed to tackle some of the most
  prominent disturbing effects found during the analyses and show how
  this set of techniques helps us to lower the average rate of
  out-of-vocabulary tokens in Twitter messages thereby improving their
  automatic Part-of-speech tagging quality.

  \keywords{twitter, social media, text normalization, spelling
    correction}
\end{abstract}
%
\section{Introduction}
When Jack Dorsey, the present CEO of Twitter Inc., was sending the
very first tweet on March 21, 2006 (Dorsey, \cite{dorsey}), he
probably could not imagine that a few years later presidents and
government officials would use this service to communicate with their
voters and the Pope would be posting short messages holding an iPad in
his hand (Pianigiani, \cite{nyt:pope}). Yet another thing that Jack
Dorsey was apparently not aware of at that moment, was the fact that
his message -- ``just setting up my twttr'' -- already contained a
word which was unknown to the majority of NLP applications existing at
that time, and that there would be many of such words in future
causing a lot of headache to natural language specialists.

Though the problem of out-of-vocabulary (OOV) words and its closely
related task of textual normalization have been extensively studied in
computational linguistics since as early as the late 1950s
(cf. Petersen, \cite{petersen}) and were certainly anything but new at
the time when mobile communication emerged, it were small messages
that revived interest in this research area in the past two
decades.

Starting from the second half of 1990-ies, the rise of SMS messages,
growing popularity of online chats and launch of social networks lead
to a significant boost in the number of publications on preprocessing
casually typed texts. But despite this increased number of researches,
the main focus of the majority of these works was on English data with
a few exceptions of works on Spanish (Oliva et al., \cite{oliva}) and
French (Beaufort et al., \cite{beaufort}).

%% One of the first scientific milestones which marked the renaissance of
%% OOV studies in CL was a comprehensive article by Karen Kukich
%% published in the renowned ACM journal on December 4-th 1992
%% \cite{kukich}. By an interesting coincidence, just exactly one day
%% before that, a British engineer called Neil Papworth had sent the
%% world's first ever SMS-message \cite{guardian:sms}. But in contrast to
%% Papworth's SMS which only said ``Merry Christmas'' to one of his
%% friends, Kukich's work comprised more than 60 pages and provided a
%% fully-fledged review of the state of the art techniques for dealing
%% with unknown words within the scope of spelling correction programs.

In the next Section we will give a short overview of recent scientific
approaches to the problem of out-of-vocabulary words in non-standard
texts. After that, in Section \ref{error:analysis} we will analyze
which types of noisiness phenomena are especially characteristic for
German Twitter. Section \ref{normalization} will subsequently describe
an automatic procedure for mitigating some of the most obvious of
those effects. In a concluding step, we will perform an evaluation of
the results of this procedure and give further suggestions for future
research.

\section{Related Work}

Before proceeding with the description of existing methods for noisy
text normalization (NTN), we first would like to define the criteria
by which these methods could be classified. It should be noted that
there already exist several classifications of NTN methods including,
for example, Kukich (\cite{kukich}), Kobus et al. (\cite{kobus}), and
Sproat et al. (\cite{sproat}). But all of these classification
attempts seem to have one common shortcoming, namely they try to
involve several independent criteria within one classification scheme.

Kukich (\cite{kukich}), for example, divided NTN techniques into six
classes:
\begin{enumerate}
\item minimum edit distance techniques;
\item similarity key techniques;
\item rule-based techniques;
\item \textit{n}-gram based techniques;
\item probabilistic techniques;
\item neural nets.
\end{enumerate}

Kobus (\cite{kobus}), on the contrary, referred to existing NTN
methods as \textit{metaphors} and split them into the following
three groups:
\begin{enumerate}
\item ``spell checking'' metaphor;
\item ``translation'' metaphor;
\item ``speech recognition'' metaphor.
\end{enumerate}

Though both divisions seem to be justified to some extent, it is
difficult to determine using them whether an NTN approach that detects
and restores incorrectly spelled words on the basis of phonetical
\textit{n}-gram statistics should fall into \textit{n}-gram based,
probabilistic, spell checking or speech recognition class.

We instead suggest using two separate types of criteria which are
unrelated to each other and characterize NTN approaches from different
points of view.

The first criterion is \emph{segmentation level}. It depends primarily
on the maximal length of syntagmatic segments which are used
\begin{inparaenum}[\itshape a\upshape)]
\item to infer in-vocabulary (IV) equivalents for OOV
  tokens\label{candidate:inferring} and
\item to choose the most probable variant\label{candidate:selection}
  among multiple possible suggestions\footnote{For the cases, when
    segments of different lengths are used for tasks
    \ref{candidate:inferring} and \ref{candidate:selection}, we note
    it explicitly in our classification on which segmentation length
    each subtask relies.}.
\end{inparaenum} For this criterion we propose division into following
classes:
\begin{enumerate}
  \item graphematic\footnote{Depending on whether phonetical
    information is involved or not at this level, this class could be
    further divided into a phonographematic and purely graphematic
    subclasses.};
  \item lexical;
  \item phrasal.
\end{enumerate}
Each broader level of this hierarchy is also supposed to either
incorporate or ignore information provided by its narrower
subsegments. In this way, we only need to mention one (the broadest)
hierarchical class for the cases when multiple segmentation levels are
involved by some techniques.

The second classification criterion regards the \emph{type of
  information induction} that was used to devise the correction
directives. This leads us to the usual NLP-taxonomy which divides all
approaches into:
\begin{enumerate}
  \item rule-based;
  \item statistical\footnote{Depending on the type of training data
    required, this class is usually divided into unsupervised,
    semi-supervised, and supervised groups.};
  \item and hybrid ones.
\end{enumerate}

Equipped with these two criteria, we could now proceed to the
description of main scientific works on NTN task which were done in
recent years.

It is worthwile noting that many approached to NTN, especially the
earlier ones, were heavily influenced by the idea of noisy channel
model. The underlying notion of this model was that observed sentences
had to be considered as corrupted versions of some original language
signals. In order to restore the undistorted signal, one had to come
up with two probabilistic submodels: the error model (one which caused
the corruption) and the language model (one which reflected the nature
of initial signal source). To these works belong Brill and Moore
(\cite{brill}), Sproat et al. (\cite{sproat}), Toutanova and Moore
(\cite{toutanova}), Clark (\cite{alexander-clark}), Choudhury et
al. (\cite{choudhury}), Cook and Stevenson (\cite{cook}), Beaufort et
al. (\cite{beaufort}) etc. The majority of these methods used either
purely graphematic (Brill and Moore, \cite{brill}; Sproat et al.,
\cite{sproat}; Clark, \cite{alexander-clark}) or phonographematic
segmentation (Toutanova and Moore \cite{toutanova}; Choudhury et al.,
\cite{choudhury}; Cook and Stevenson, \cite{cook}) for error
models. For language model, normally phrasal segments (word
\textit{n}-grams) were used. From the point of view of information
induction almost all of these techniques could be characterized as
supervised statistical, with the exception of Cook and Stevenson
(\cite{cook}) who claimed to use an unsupervised technique.

The raising influence and improved quality of machine translation
tools and applications in the 2000s lead to the development of NTN
technologies which used broader levels of segmentation. In \cite{aw},
Aw et al. suggested a supervised statistical system for normalization
of SMS-messages. Their system operated exclusively on phrasal level. A
few years later, Clark and Araki (\cite{clark-araki}) described a
purely rule-based phrasal technique.

As stated by Kobus et al. (\cite{kobus}), NTN methods relying on
either graphematic or phrasal segments usually revealed complementary
strengths and weaknesses. This notion led NLP scientists to the idea
that incorporating multiple levels of language in one NTN system could
improve the total quality of the system as a whole since different
levels would benefit from each other. As a consequence of this notion,
a wealth of combined techniques emerged in the past few years. Among
these we should especially mention Kobus et al. (\cite{kobus}),
Kaufmann (\cite{kaufmann}), Han and Baldwin (\cite{han}), and Oliva et
al. (\cite{oliva}). The system suggested by Kobus et
al. (\cite{kobus}), for example, used a hybrid phrasal approach in a
pre-processing step and subsequently fed the output of this
pre-processing into a finite state transducer (FST). The FST performed
a phonographematic segmentation of data and derived normalization
equivalents on the basis of statistical inference. Another approach
proposed by Kaufmann (\cite{kaufmann}) first used unambigous lexical
mappings and straightforward graphematic correction rules to reduce
the noisiness effects and then redirected pre-normalized input to a
statistical phrasal MT system.

Eventually, in \cite{han}, an article called ``Lexical Normalization
of Short Text Messages: Makn Sens a \#{}twitter'' was published by Han
and Baldwin. In this article the authors separated the tasks of
identification of ill-formed words and finding appropriate correction
for them. For the former problem, they first generated a confusion set
(CS) for each word unknown to \texttt{GNU aspell}. Based on this set,
the decision was made whether a particular word had to be corrected or
regarded as vlid. Subsequently, for words identified as ill-formed the
most probable restoration candidate was chosen from CS by combining
features resulting from dictionary lookup, analysis of surrounding
context, and estimating word similarity to each proposed correction
variant. According to authors' estimations, this combination allowed
them to outperform most of the NTN methods existing at that time.

In our next section will also perform quantitive and qualitative
analyses of unknown words in German Twitter texts in order to assess
whether and to what extent the problem of ill-formed words is relevant
for them. Relying on these analyses we then will suggest a set of
ad-hoc techniques which could help one lower the average rate of
incorrectly spelled words in this kind of text.

\section{Analysis of Unknown Tokens}\label{error:analysis}

In order to estimate the percentage of unknown words in Twitter
messages, we randomly selected 10,000 tweets from a previously
collected corpus, split them into sentences and tokenized using social
media aware tokenizer by Christopher Potts
\footnote{\url{http://sentiment.christopherpotts.net/code-data/happyfuntokenizing.py}}.
After skipping all words which did not contain any alphabetic
character or consisted only of a single letter, we
obtained a list of 129,146 tokens. As reference systems for dictionary
lookup we used open source spell checking program
\texttt{hunspell}\footnote{Ispell Version 3.2.06 (Hunspell Version
  1.3.2); dictionary de\_DE.} and publicly available part-of-speech
tagger \texttt{TreeTagger}\footnote{Version 3.2 with German parameter
  file UTF-8.} (Schmid, \cite{schmid}).

Out of this token list, 26,018 tokens (20.15~\%) were regarded as
unknown by \texttt{hunspell} and 28,389 tokens (21.98~\%) were
considered as OOV by \texttt{TreeTagger}. We also performed similar
estimations after leaving only unique words without taking into
account their frequencies. This allowed us to shrink our initial token
list by four times to 32,538 unique tokens. The relative rate of
unknown words raised as expected and run up to 46.96~\% for
\texttt{hunspell} and 58.24~\% for \texttt{TreeTagger}.

Here once again the question of classification arose but this time
with regard to the reasons, why different tokens could have been
omitted from corresponding applications' dictionaries. In this
respect, division into following groups seemed to be appropriate for
us:
\begin{enumerate}
  \item \textbf{Objective limitedness of machine-readable dictionary
    (MRD)}. To this group we counted words of basic vocabulary which
    did not get into applications' MRD either because they supposedly
    were rare or because they did not exist at the time when
    dictionaries were created. Another reason for inclusion in this
    type was the belonging of a word to an open lexical or
    part-of-speech class (like, for example, named entities or
    interjections) which are usually deliberately not included in MRDs
    due to impossibility to cover these classes fully;\label{dict}
  \item \textbf{Unintended sloppiness of user's input}. In the scope
    of this group we considered unintended typos including unexpected
    truncation of words at the end of Twitter messages\footnote{As is
      generally known, Twitter imposes a strong restriction on the
      length of posted messages which can be no longer than 140
      characters. Upon exceeding this length, tweets get automatically
      truncated.};\label{spell}
  \item \textbf{Stylistic specifics of text genre}. This group
    comprised words which could be considered as illegal from the
    point of view of standard language texts but were perfectly valid
    terms in the domain of web discourse and Twitter
    communication.\label{style}
\end{enumerate}

In order to see how out-of-vocabulary words were distributed among and
within these 3 major groups, we manually analyzed all OOV tokens which
appeared in text more than once and also looked at 1,000 randomly
selected hapax legomena. The results of these estimations are shown
and explained below.

We subdivided group \ref{dict} into the following subgroups:
\begin{enumerate}
\item regular German words, e.g. \textit{aufm}, \textit{losziehen};\label{regular}
\item compounds, e.g.  \textit{Altwein}, \textit{Amtsapothekerin};\label{compound}
\item abbreviations, e.g. \textit{NBG}, \textit{OL};\label{abbr}
\item interjections, e.g.  \textit{aja}, \textit{haha};\label{inj}
\item named entities, with subclasses:\label{ne}
  \begin{enumerate}
  \item persons, e.g.  \textit{Ahmadinedschad}, \textit{Schweiger};
  \item geographic locations, e.g.  \textit{Biel}, \textit{Limmat};
  \item companies, e.g. \textit{Apple}, \textit{Facebook};
  \item product names, e.g. \textit{iPhone}, \textit{MacBook};
  \end{enumerate}
\item neologisms, with subclasses:\label{neolog}
  \begin{enumerate}
    \item newly coined German terms, e.g. \textit{entfolgen},
      \textit{geskypt};\label{new}
    \item loanwords, e.g. \textit{Community},
      \textit{Stream};\label{loan}
  \end{enumerate}
\item and, finally, foreign words like \textit{is} or \textit{now}
  which in contrast to \ref{loan} were not mentioned in any existing
  German lexicons nor they complied with inflectional rules of German
  grammar.\label{fw}
\end{enumerate}
Though this division is admittedly arbitrarily to a certain degree and
involves different linguistic criteria simultaneously, the underlying
notion for it was simple. Valid words could have been omitted from an
MRD either due to the limitation of developers' capacities (group
\ref{regular}), active word formation or lexical productivity of the
language (groups \ref{compound} through \ref{new}) or also due to
language's openness to foreign language systems (groups \ref{loan} and
\ref{fw}).

In Table \ref{table:mrd} percentage figures for each of the above
subgroups are shown. We have considered OOV-distribution for both
\texttt{hunspell} and \texttt{TreeTagger}. For both of them we
estimated the percentage of particular subclass with regard to the
total number of occurrences of OOV-tokens (column ``\totaloov{}'') as
well as with regard to their percentage in the list of unique OOVs
(column ``\uniqoov{}'') without taking into account their frequencies.

\begin{table}
  \caption{Distribution of OOV words belonging to the class
    ``objective limitedness of MRD''\label{table:mrd}}
  \begin{tabular}{p{0.4\textwidth}*{4}{>{\centering\arraybackslash}p{0.15\textwidth}}}
    \hline\noalign{\smallskip}
    \multirow{2}{*}{OOV subclass} & %
    \multicolumn{2}{c}{\texttt{hunspell}} & %
    \multicolumn{2}{c}{\texttt{TreeTagger}}\\
    & \totaloov{} & \uniqoov{} & \totaloov{} & \uniqoov{}\\
    \noalign{\smallskip} \hline
    regular German words & 7.88 & 8.85 &  & \\
    compounds & 1.22 & 2.42 &  & \\
    abbreviations & 4.12 & 4.91 &  & \\
    interjections & 5.93 & 4.47 &  & \\
    person names & 4.79 & 6.45 &  & \\
    geographic locations & 1.51 & 2.53 &  & \\
    company names & 2.26 & 2.8 &  & \\
    product names & 2.08 & 2.49 &  & \\
    newly coined terms & 1.22 & 1.24 &  & \\
    loanwords & 3.72 & 4.05 &  & \\
    foreign words & 11.68 & 14 &  & \\\hline
    {\bfseries total} & 46.41 & 54.21 &  & \\
    \noalign{\smallskip} \hline
  \end{tabular}
\end{table}

Similarly to class \ref{dict} we subdivided the group of misspellings
into the following subgroups:
\begin{enumerate}
  \item insertion;
  \item deletion;
  \item substitution;
\end{enumerate}
according to the kind of operation which led to a particular spelling
mistake. The statistics on distribution of these subgroups is shown in
Table \ref{table:spell}.
\begin{table}
  \caption{Distribution of OOV words belonging to the class
    ``unintended sloppiness of user's input''\label{table:spell}}
  \begin{tabular}{p{0.4\textwidth}*{4}{>{\centering\arraybackslash}p{0.15\textwidth}}}
    \hline\noalign{\smallskip}
    \multirow{2}{*}{OOV subclass} & %
    \multicolumn{2}{c}{\texttt{hunspell}} & %
    \multicolumn{2}{c}{\texttt{TreeTagger}}\\
    & \totaloov{} & \uniqoov{} & \totaloov{} & \uniqoov{}\\
    \noalign{\smallskip} \hline
    insertion &  &  &  & \\
    deletion &  &  &  & \\
    substitution &  &  &  & \\\hline
    {\bfseries total} &  &  &  & \\
    \noalign{\smallskip} \hline
  \end{tabular}
\end{table}


\begin{table}
  \caption{Distribution of OOV words belonging to the class
    ``stylistic specifics of text genre''}
  \begin{tabular}{p{0.4\textwidth}*{4}{>{\centering\arraybackslash}p{0.15\textwidth}}}
    \hline\noalign{\smallskip}
    \multirow{2}{*}{OOV subclass} & %
    \multicolumn{2}{c}{\texttt{TreeTagger}} & %
    \multicolumn{2}{c}{\texttt{hunspell}}\\
    & \totaloov{} & \uniqoov{} & \totaloov{} & \uniqoov{}\\
    \noalign{\smallskip} \hline
    hashtags &  &  &  & \\
    @-tokens &  &  &  & \\
    links &  &  &  & \\
    smileys &  &  &  & \\
    slang &  &  &  & \\\hline
    {\bfseries total} &  &  &  & \\
    \noalign{\smallskip} \hline
  \end{tabular}
\end{table}

\section{Text Normalization Procedure}\label{normalization}
\subsection{Replacement of Twitter-Specific Phenomena}
%
\subsection{Restoration of Umlauts}
%
\subsection{Squeezing of Elongated Words}
%
\subsection{Translation of Slang Idioms}
%
\section{Evaluation}\label{evaluation}


\section{Conclusion}\label{concl}

This article provided an overview of exisiting approaches to noisy
text normalization task. Additionally, all mentioned methods were
classified on the basis of two independent criteria. In section
\ref{error:analysis}, we performed qulitative and quantitive analyses
of out-of-vocabulary words in German tweets and suggested a set of
ad-hoc techniques for mitigating their potential negative influence on
natural language processing. This procedure allowed us to reduce the
total OOV rate by ... \% for \texttt{hunspell} and by ... \% for
\texttt{TreeTagger}.

Nevertheless, we should honestly admit that our system still has
potential for development and research, since it mainly addresses only
one of three main groups of OOV tokens. Future directions should
certainly include a more thorough tackling of unintentional spelling
mistakes and especially their most prominent types -- deletions and
substitutions.

Furthemore, a better evalution technique as well as comparison with
other systems are needed for our normalization procedure. On the one
hand, an \textit{extrinsic} evaluation should be performed (cf. Sparck
Jones and Galliers, \cite{sparck}) which means that we not only have
to show how the rate of OOV words goes down but much more how this
lower OOV rate affects the work of the whole NLP system. On the other
hand, we need to assess the quality of our procedure's work on the
basis of metrics used by other researchers.

One possible estimation criterion which was used by Aw et
al. (\cite{aw}), Kaufmann (\cite{kaufmann}), Beaufort et
al. (\cite{beaufort}), and Oliva et al. (\cite{oliva}) is the BLEU
score (Papineni et al., \cite{papineni}). Another possibility would be
to use the Word (WER) and Sentence Error Rates (SER) as suggested by
Kobus et al. (\cite{kobus}). However, an obvious difficulty that we
already encountered here is that both metrics highly rely on a
subjective notion of the look of a ``normalized'' message. While the
BLEU score could be an approriate criterion for normalization of
messages like ``i luv ma mather and wd do evrythin 4 her'' which in
fact looks like as if it were not English. But such highly distorted
tweets are rather atypical for German. In this regard, WER and SER
could be considered as more appropriate measurement criteria. But
these metrics once again are rather dealing with spelling mistakes and
would highly depend on whether one, for example, would consider the
hash sign in a hashtag as an error.

\subsection*{\ackname}

This work was financially supported by ... as part of collaborative
project ``...''. The authors are also thankful to ... for help with
the analysis of data.

%
% ---- Bibliography ----
%
\bibliographystyle{splncs}
\begin{thebibliography}{}
\bibitem[2006]{aw}
  Aw, A., Zhang, M., Xiao, J., Su, J.: %
  A Phrase-based Statistical Model for {SMS} Text Normalization. %
  COLING/ACL (2006) 33--40 %

\bibitem[2002]{bangalore}
  Bangalore, S., Murdock, V., Riccardi, G.: %
  Bootstrapping Bilingual Data using Consensus Translation for a %
  Multilingual Instant Messaging System. %
  COLING (2002) 33--40 %

\bibitem [2010]{beaufort}
  Beaufort, R., Roekhaut, S., Cougnon, L. A.,  Fairon, C.: %
  A Hybrid Rule/Model-Based Finite-State Framework for %
  Normalizing SMS Messages. %
  ACL (2010) 770--779 %

\bibitem[2000]{brill}
  Brill, E., Moore, R. C.: %
  An improved model for noisy channel spelling correction. %
  ACL %
  (2000) 286--293 %

\bibitem[2003]{alexander-clark}
  Clark, A.: %
  Pre-processing very noisy text. %
  In Proceedings of Workshop on Shallow Processing of Large Corpora. %
  (2003) 12--22

\bibitem[2007]{choudhury}
  Choudhury, M., Saraf, R., Jain, V., Mukherjee, A., Sarkar, S., Basu %
  A.: %
  Investigation and Modeling of the Structure of Texting Language. %
  International Journal of Document Analysis and Retrieval: Special %
  Issue on Analytics of Noisy Text. {\bfseries 10} %
  (2007) 157--174 %

\bibitem[2011]{clark-araki}
  Clark, E., Araki, K.: %
  Text Normalization in Social Media: Progress, Problems and %
  Applications for a Pre-processing System of Casual English. %
  PACLING. Procedia - Social and Behavioral Sciences {\bfseries 27} %
  (2011) 2--11

\bibitem[2009]{cook}
  Cook, P., Stevenson, S.: %
  An unsupervised model for text message normalization, %
  Proceedings of the Workshop on Computational Approaches %
  to Linguistic Creativity. %
  CALC '09. (2009) 71--78

\bibitem[2006]{dorsey}
  Dorsey, J.: %
  "just setting up my twttr". %
  \url{https://twitter.com/jack/status/20} %
  Accessed February 26, 2013. (2006)

\bibitem[2011]{han}
  Han, B., Baldwin, T.: %
  Lexical Normalization of Short Text Messages: Makn Sens %
  a \#{}twitter. %
  ACL HLT. %
  (2011) 368--378

\bibitem[2008]{jurafsky}
  Jurafsky, D., Martin, J. H.: %
  Speech and Language Processing. 2nd Edition. %
  Prentice Hall (2008) 129, 323

\bibitem [2010]{kaufmann}
  Kaufmann, M.: %
  Syntactic normalization of twitter messages. %
  The 8-th International Conference on Natural Language Processing. %
  (2010)

\bibitem [2008]{kobus}
  Kobus, C., Yvon, F., Damnati, G.: %
  Normalizing {SMS}: are Two Metaphors Better than One? %
  COLING (2008) 441--448

\bibitem[1992]{kukich}
  Kukich, K.: %
  Techniques for Automatically Correcting Words in Text. %
  ACM Computing Surveys {\bfseries 24/4} (1992) 378--439

\bibitem[2012]{guardian:sms}
  McVeigh, T.: %
  Text messaging turns 20. %
  The Observer (December 1, 2012)

\bibitem[2012]{mukherjee}
  Mukherjee, S., Malu, A., Balamurali, A. R., Bhattacharyya, P.: %
  TwiSent: A Multistage System for Analyzing Sentiment in Twitter. %
  In Proceedings of The 21st ACM Conference on Information and %
  Knowledge Management CIKM 2012, Hawai, (Oct 29 - Nov 2, 2012)


  McVeigh, T.: %
  Text messaging turns 20. %
  The Observer (December 1, 2012)

\bibitem [2013]{oliva}
  Oliva, J., Serrano, J. I., and Del Castillo, M. D., %
  and Igesias, �.: %
  A {SMS} normalization system integrating multiple %
  grammatical resources. %
  Natural Language Engineering. %
  (2013) 121--141 %

\bibitem [2002]{papineni}
  Papineni, K., Roukos, S., Ward, T., Zhu, W.-J.: %
  Bleu: a Method for Automatic Evaluation of Machine Translation. %
  ACL (2002) 311--318

\bibitem [1980]{petersen}
  Petersen, L. J.: %
  Computer Programs for Detecting and Correcting Spelling Errors. %
  Communications of the ACM {\bfseries 23/ 12} (1980) 676--687

\bibitem[2012]{nyt:pope}
  Pianigiani, G., Donadio, R.: %
  Twitter Has A New User: The Pope. %
  The New York Times. Page A6. (December 4, 2012)

\bibitem[2001]{sproat}
  Sproat, R., Black, A. W., Chen, S. F., Kumar, S., %
  Ostendorf, M., Richards, Ch.: %
  Normalization of non-standard words. %
  Computer Speech \& Language, {\bfseries 15/3} (2001) 287--333

\bibitem[1994]{schmid}
  Schmid, H.: %
  Probabilistic Part-of-Speech Tagging Using Decision Trees. %
  In Proceedings of International Conference on New Methods in %
  Language Processing. (1994)

\bibitem[1996]{sparck}
  Sparck Jones, K., Galliers, J. R.: %
  Evaluating Natural Language Processing Systems. %
  An Analysis and Review. Lecture Notes in Computer %
  Science 1083, Springer. (1996)

\bibitem[2012]{terdiman}
  Terdiman, D.: %
  Report: Twitter hits half a billion tweets a day. %
  \url{http://timmurphy.org/2009/07/22/line-spacing-in-latex-documents/},%
  Accessed February 26, 2013. (2012)

\bibitem[2002]{toutanova}
  Toutanova, K., Moore, R. C.: %
  Pronunciation Modeling for Improved Spelling Correction. %
  ACL (2002) 144--151

\end{thebibliography}
%
% ---- End of Bibliography ----
%
\end{document}
