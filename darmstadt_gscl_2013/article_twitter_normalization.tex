\documentclass[citeauthoryear]{llncs}

%%%%%%%%%%%%%%%%%%%%%%%%%%%%%%%%%%%%%%%%%%%%%%%%%%%%%%%%%%%%%%%%%%
\usepackage{makeidx}  % allows for indexgeneration
\usepackage{paralist} % for inparaenum
\usepackage{multirow} % for multirow
\usepackage{tabularx} % for centering in table
\usepackage{url}

%%%%%%%%%%%%%%%%%%%%%%%%%%%%%%%%%%%%%%%%%%%%%%%%%%%%%%%%%%%%%%%%%%
\newcommand{\totaloov}{\% of total OOVs}
\newcommand{\uniqoov}{\% of unique OOVs}

%%%%%%%%%%%%%%%%%%%%%%%%%%%%%%%%%%%%%%%%%%%%%%%%%%%%%%%%%%%%%%%%%%
\begin{document}
\pagestyle{headings}  % switches on printing of running heads
\title{Rule-Based Normalization of German Twitter Messages}
\titlerunning{Rule-Based Normalization of German Twets}  % abbreviated title (for running head)
%                                     also used for the TOC unless
%                                     \toctitle is used
%
\author{Uladzimir Sidarenka \and Tatjana Scheffler \and Manfred Stede}
%
\authorrunning{Sidarenka et al.} % abbreviated author list (for running head)
%
%%%% list of authors for the TOC (use if author list has to be modified)
\tocauthor{Uladzimir Sidarenka, Tatjana Scheffler, Manfred Stede}
%
\institute{University of Potsdam,\\
  \email{{uladzimir.sidarenka@uni-potsdam.de,
      tatjana.scheffler@uni-potsdam.de,
      manfred.stede@uni-potsdam.de},\\ WWW home page:
    \texttt{http://www.ling.uni-potsdam.de/acl-lab/SocMedia/main.htm}}}

\maketitle              % typeset the title of the contribution

\begin{abstract}
  %% Since its launch in March 2006 and until today Twitter constantly
  %% gained more and more popularity as a communication means among the
  %% Internet users. As of the end of 2012, approximately half a billion
  %% messages were published every day by using its services (Terdiman,
  %% \cite{terdiman}).  But, unfortunately, this abundant resource of
  %% textual information can not be exploited to the full extent by
  %% solely relying on standard NLP frameworks, because misspellings,
  %% active usage of slang, and various other aspects of texting language
  %% significantly decrease the chances of successful automatic analysis
  %% of this kind of data.

  This article gives an overview of existing approaches to the problem
  of out-of-vocabulary (OOV) tokens and noisiness phenomena in natural
  language texts. These approaches are classified with regard to the
  size of text spans and knowledge inference mechanisms which they
  rely on in their work. Additionally to that, we conduct quantitative
  and qualitative analyses of unknown words in German Twitter
  messages, in order to see how relevant the OOV and text
  normalization problems are for this particular kind of micro-texts
  and what the characteristics of these OOVs are. In a concluding
  step, we present a set of ad-hoc techniques which are supposed to
  tackle some of the most prominent disturbing effects found during
  the analyses and show how this set of techniques helps us lower the
  average rate of out-of-vocabulary tokens in Twitter messages and how
  this lower OOV-rate in turn helps improve the quality of automatic
  part-of-speech tagging.

  \keywords{twitter, social media, text normalization, spelling
    correction}
\end{abstract}
%
\section{Introduction}
When Jack Dorsey, the present CEO of Twitter Inc., was sending the
very first tweet on March 21, 2006 (Dorsey, \cite{dorsey}), he
probably could not imagine that a few years later presidents and
government officials would use this service to communicate with their
voters and the Pope would be posting short messages holding an iPad in
his hand (Pianigiani, \cite{nyt:pope}). Yet another thing that Jack
Dorsey was apparently not aware of at that moment, was the fact that
his message -- ``just setting up my twttr'' -- already contained a
word which was unknown to the majority of NLP applications existing at
that time, and that there would be many of such words in future
causing a lot of headache to natural language specialists.

Though the problem of out-of-vocabulary words and its closely related
task of textual normalization have been extensively studied in
computational linguistics since as early as the late 1950s
(cf. Petersen, \cite{petersen}) and were certainly anything but new at
the time when mobile communication emerged, it were small messages
that revived interest in this research area in the past two decades.

%% Starting from the second half of the 1990s, not only the number of
%% publications on automatic text correction gradually increased every
%% year but also the focus of those researches steadily turned away from
%% normalization of manually typed or OCR processed official documents to
%% the processing of sloppily created user messages and online chat
%% posts. But even despite this increased interest, most of the
%% researches still were dealing with only English text data. A few
%% exceptions from that are works on French (Beaufort et al.,
%% \cite{beaufort}) and Spanish (Oliva et al., \cite{oliva}).

%% One of the first scientific milestones which marked the renaissance
%% of OOV studies in CL was a comprehensive article by Karen Kukich
%% published in the renowned ACM journal on December 4-th 1992
%% \cite{kukich}. By an interesting coincidence, just exactly one day
%% before that, a British engineer called Neil Papworth had sent the
%% world's first ever SMS-message \cite{guardian:sms}. But in contrast
%% to Papworth's SMS which only said ``Merry Christmas'' to one of his
%% friends, Kukich's work comprised more than 60 pages and provided a
%% fully-fledged review of the state of the art techniques for dealing
%% with unknown words within the scope of spelling correction
%% programs.

%% In this article we are going to provide some more insight about the
%% relevance of OOV issues for German language by showing how many
%% out-of-vocabulary tokens are found in German online texts on average
%% and what kind of linguistic phenomena are main sources of these
%% OOVs. The next Section will first give a short overview and provide a
%% classification of recent scientific approaches to the problem of
%% out-of-vocabulary words in non-standard texts. After that, in Section
%% \ref{error:analysis} we will analyze which types of noisiness
%% phenomena are especially common in German Twitter. Section
%% \ref{normalization} will subsequently describe an automatic procedure
%% for mitigating some of the most prominent of those effects. In a
%% concluding step, we will perform an evaluation of the results of this
%% procedure and give some further suggestions for future research.

In the next Section, we will give a short overview of existing
scientific approaches to the problem of tackling out-of-vocabulary
words in non-standard texts. After that, in Section
\ref{error:analysis} we will analyze which types of noisiness
phenomena are especially characteristic for German Twitter. Section
\ref{normalization} will subsequently describe an automatic procedure
for mitigating some of the most prominent of those effects. In a
concluding step, we will perform an evaluation of the results of this
procedure and give further suggestions for future research.

\section{Related Work}

Before proceeding with the description of recent methods for noisy
text normalization (NTN), we first would like to define the criteria
by which these methods could be classified. It should be noted that
there already exist several classifications of NTN approaches
including, for example, Kukich (\cite{kukich}) and Kobus et
al. (\cite{kobus}).

Kukich (\cite{kukich}), for instance, divided all NTN techniques into
six classes:
\begin{enumerate}
\item minimum edit distance techniques;
\item similarity key techniques;
\item rule-based techniques;
\item \textit{n}-gram based techniques;
\item probabilistic techniques;
\item neural nets.
\end{enumerate}
Kobus (\cite{kobus}), on the contrary, referred to NTN methods as
\textit{metaphors} and split them into the following three groups:
\begin{enumerate}
\item ``spell checking'' metaphor;
\item ``translation'' metaphor;
\item ``speech recognition'' metaphor.
\end{enumerate}

Though both of these divisions seem to be justified to some extent, it
is difficult to determine using them whether an NTN approach that
detects and restores incorrectly spelled words on the basis of
phonetical \textit{n}-gram statistics should fall into the
\textit{n}-gram based, probabilistic, spell checking or speech
recognition class.

This confusion is explained by the fact that the above classifications
both rely on several independent criteria at the same time, however
each of these criteria characterizes an NTN system from a different
point of view. As a consequence of this, unambiguous assignment of an
NTN approach to one particular class often becomes impossible. In
order to avoid this, we instead suggest using separate classifications
for each type of involved characterizing criterion.

One of such criteria which in our opinion would be worth a separate
classification is \emph{segmentation level} that is used
\begin{inparaenum}[\itshape a\upshape)]
\item to infer in-vocabulary (IV) equivalents for OOV
  tokens\label{candidate:inferring} and
\item to choose the most probable variant\label{candidate:selection}
  among multiple possible suggestions\footnote{For the cases, when
    segments of different lengths are used for tasks
    \ref{candidate:inferring} and \ref{candidate:selection}, we note
    it explicitly in our classification on which segmentation length
    each of these subtasks relies.}.
\end{inparaenum} For this criterion, we propose division into the following
classes:
\begin{enumerate}
  \item graphematic\footnote{Depending on whether phonetical
    information is involved or not at this level, this class could be
    further divided into a phonographematic and purely graphematic
    subclasses.};
  \item lexical;
  \item phrasal.
\end{enumerate}
Each broader level of this hierarchy is supposed to either incorporate
or ignore information provided by its narrower subsegments. In this
way, we only need to mention one (the broadest) hierarchical class for
the cases when multiple segmentation levels are involved by some
techniques.

The second criterion regards the \emph{type of information induction}
that is used to devise the correction rules. This leads us to the
usual NLP-taxonomy which divides all approaches into:
\begin{enumerate}
  \item rule-based;
  \item statistical\footnote{Depending on the type of training data
    required, this class is in turn usually divided into unsupervised,
    semi-supervised, and supervised groups.};
  \item and hybrid ones.
\end{enumerate}

According to these two classification criteria, recent approaches to
the NTN task could be grouped together as follows.

The earlier works on NTN commonly relied on either purely graphematic
or phonographematic levels of segmentation for deriving normalization
variants of incorrectly spelled words. To purely graphematic systems
belong methods suggested by Brill and Moore (\cite{brill}), Sproat et
al. (\cite{sproat}), and Clark (\cite{alexander-clark}). As
phonographematic approaches one could regard works done by Toutanova
and Moore (\cite{toutanova}), Choudhury et al. (\cite{choudhury}),
Cook and Stevenson (\cite{cook}), Han and Baldwin (\cite{han})
etc. With regard to the type of information inference, most of these
methods were supervised with the exceptions of Cook and Stevenson
(\cite{cook}) and Han and Baldwin (\cite{han}) who claimed to use an
unsupervised technique. Sproat et al. (\cite{sproat}) described in
their article both a supervised and unsupervised approach.

Starting from the second half of the 2000s, the raising influence and
improved quality of machine translation tools led to the development
of NTN technologies which used broader levels of segmentation. In
\cite{aw}, Aw et al. suggested a supervised statistical system for
normalization of SMS-messages which operated on automatically aligned
phrases A few years later, Clark and Araki (\cite{clark-araki})
described a purely rule-based method which used mappings of
non-standard words and phrases to their corresponding standard
language forms.

As noted by Kobus et al. (\cite{kobus}), NTN methods relying on either
graphematic or phrasal segments usually revealed complementary
strengths and weaknesses. This notion led NLP scientists to the idea
that by incorporating multiple levels of the language into one NTN
system the total performace of the whole system would improve as
different sources of information would benefit from each other. As a
consequence of this, a wealth of combined techniques emerged in the
past few years. Among these we should especially mention works by
Kobus et al. (\cite{kobus}), Kaufmann (\cite{kaufmann}), and Oliva et
al. (\cite{oliva}). The majority of these systems used the whole range
of segmentation levels from phonographematatic to phrasal one, and in
many cases they also applied different knowledge inference mechanisms
to different levels of the language.

%% Eventually, in \cite{han}, an article called ``Lexical Normalization
%% of Short Text Messages: Makn Sens a \#{}twitter'' was published by Han
%% and Baldwin. In this article the authors separated the tasks of
%% identification of ill-formed words and finding appropriate correction
%% for them. For the former problem, they first generated a confusion set
%% (CS) for each word unknown to \texttt{GNU aspell}. Based on this set,
%% the decision was made whether a particular word had to be corrected or
%% regarded as vlid. Subsequently, for words identified as ill-formed the
%% most probable restoration candidate was chosen from CS by combining
%% features resulting from dictionary lookup, analysis of surrounding
%% context, and estimating word similarity to each proposed correction
%% variant. According to authors' estimations, this combination allowed
%% them to outperform most of the NTN methods existing at that time.

It should however be noted that almost all of the above methods mainly
concentrated on only English data. A few exceptions from that are
approaches suggested by Beaufort et al. (\cite{beaufort}) for French,
and Oliva et al. (\cite{oliva}) for Spanish. To find out which
peculiarities of ill-formed words are characteristic for German, we
will perform a quantitive and qualitative analysis of unknown words in
German Twitter in the next Section in order to see what kind of NTN
techniques would be most suitable for handling such words there.

\section{Analysis of Unknown Tokens}\label{error:analysis}

In order to estimate the percentage of unknown words in Twitter
messages, we randomly selected 10,000 tweets from a previously
collected corpus, split them into sentences and tokenized using social
media-aware tokenizer by Christopher Potts
\footnote{\url{http://sentiment.christopherpotts.net/code-data/happyfuntokenizing.py}}.
After skipping all words which did not contain any alphabetic
characters or consisted only of a single letter, we obtained a list of
129,146 tokens. As reference systems for dictionary lookup we used
open-source spell checking program \texttt{hunspell}\footnote{Ispell
  Version 3.2.06 (Hunspell Version 1.3.2); dictionary de\_DE.} and
publicly available part-of-speech tagger
\texttt{TreeTagger}\footnote{Version 3.2 with German parameter file
  UTF-8.} (Schmid, \cite{schmid}).

Out of this token list, 26,018 tokens (20.15~\%) were regarded as
unknown by \texttt{hunspell} and 28,389 tokens (21.98~\%) were
considered as OOV by \texttt{TreeTagger}. We also performed similar
estimations after leaving only unique words without taking into
account their frequencies. This allowed us to shrink our initial token
list by four times to 32,538 unique tokens. The relative rate of
unknown words raised as expected and run up to 46.96~\% for
\texttt{hunspell} and 58.24~\% for \texttt{TreeTagger}.

We classified found OOV tokens into the following three groups
according to the reasons why these tokens could have been omitted from
corresponding applications' dictionaries:
\begin{enumerate}
  \item \textbf{Objective limitedness of machine-readable dictionaries
    (MRD)}. Among this group, we counted words of basic vocabulary
    which did not get into applications' MRD either because they
    supposedly were rare or did not exist at the time when
    dictionaries were created. Another reason for inclusion in this
    type was the belonging of a word to an open lexical or
    part-of-speech class (like, for example, named entities or
    interjections) which are often omitted from MRDs due to the
    impossibility to fully cover them;\label{dict}
  \item \textbf{Sloppiness of users' input}. In the scope of this
    second group, we considered incorrect spellings of words
    encountered in text;\label{spell}
  \item \textbf{Stylistic specifics of text genre}. This group
    comprised words which could be considered as illegal from the
    point of view of standard language but were perfectly valid terms
    in the domain of web discourse or more specifically in Twitter
    communication.\label{style}
\end{enumerate}

In order to see how detected out-of-vocabulary words were distributed
among and within these 3 major groups, we manually analyzed all OOV
tokens which appeared in text more than once and also looked at 1,000
randomly selected hapax legomena. The results of these estimations are
shown and explained below.

We subdivided class \ref{dict} into the following subclasses:
\begin{enumerate}
\item regular German words, e.g. \textit{Piraterie},
  \textit{losziehen};\label{regular}
\item compounds, e.g.  \textit{Altwein},
  \textit{Amtsapothekerin};\label{compound}
\item abbreviations, e.g. \textit{NBG}, \textit{OL};\label{abbr}
\item interjections, e.g.  \textit{aja}, \textit{haha};\label{inj}
\item named entities, with subclasses:\label{ne}
  \begin{enumerate}
  \item persons, e.g.  \textit{Ahmadinedschad}, \textit{Schweiger};
  \item geographic locations, e.g.  \textit{Biel}, \textit{Limmat};
  \item companies, e.g. \textit{Apple}, \textit{Facebook};
  \item product names, e.g. \textit{iPhone}, \textit{MacBook};
  \end{enumerate}
\item neologisms, with subclasses:\label{neolog}
  \begin{enumerate}
    \item newly coined German terms, e.g. \textit{entfolgen},
      \textit{gegoogelt};\label{new}
    \item loanwords, e.g. \textit{Community},
      \textit{Stream};\label{loan}
  \end{enumerate}
\item and, finally, foreign words like \textit{is} or \textit{now}
  which in contrast to \ref{loan} were not mentioned in any existing
  German lexica and did not comply with inflectional rules of German
  grammar.\label{fw}
\end{enumerate}
Though this division is admittedly arbitrary to a certain degree and
also has the disadvantage of simultaneously involving different
linguistic criteria, the underlying notion here was simple -- valid
words could have been omitted from an MRD either due to the
limitations of developers' capacities (group \ref{regular}), active
word formation processes or lexical productivity of the language
itself (groups \ref{compound} through \ref{new}) or also due to
language's openness to foreign language systems (groups \ref{loan} and
\ref{fw}).

In Table \ref{table:mrd}, percentage figures for each of the above
subgroups are shown. We have considered OOV-distributions for both
\texttt{hunspell} and \texttt{TreeTagger}. For each of them, we
estimated the percentage of a particular subclass with regard to the
total number of occurrences of all OOV-tokens (column ``\totaloov{}'')
as well as with regard to their percentage rate in the list of only
unique OOVs disregarding frequencies (column ``\uniqoov{}'').
\begin{table}
  \caption{Distribution of OOV words belonging to the class
    ``Objective limitedness of MRD''\label{table:mrd}}
  \begin{tabular}{p{0.4\textwidth}*{4}{>{\centering\arraybackslash}p{0.15\textwidth}}}
    \hline\noalign{\smallskip}
    \multirow{2}{*}{OOV subclass} & %
    \multicolumn{2}{c}{\texttt{hunspell}} & %
    \multicolumn{2}{c}{\texttt{TreeTagger}}\\
    & \totaloov{} & \uniqoov{} & \totaloov{} & \uniqoov{}\\
    \noalign{\smallskip} \hline
    regular German words & 7.82 & 8.78 & 2.8 & 3.49\\
    compounds & 1.21 & 2.42 & 2.51 & 4.55\\
    abbreviations & 3.98 & 4.77 & 3.27 & 3.44\\
    interjections & 5.95 & 4.54 & 5.58 & 4.29\\
    person names & 4.73 & 6.41 & 2.32 & 3.47\\
    geographic locations & 1.5 & 2.53 & 1.16 & 1.88\\
    company names & 2.27 & 2.84 & 4.35 & 3.01\\
    product names & 2.13 & 2.57 & 2.45 & 3.23\\
    newly coined terms & 1.35 & 1.31 & 3.33 & 2.38\\
    loanwords & 3.68 & 4.03 & 3.29 & 2.87\\
    foreign words & 11.5 & 13.76 &  9.57 & 10.91\\\hline
    {\bfseries total} & 46.12 & 53.96 & 40.63 & 43.52\\
    \noalign{\smallskip} \hline
  \end{tabular}
\end{table}

Similarly to class \ref{dict}, we subdivided the group ``Sloppiness of
users' input'' into the following subgroups:
\begin{enumerate}
  \item insertions, e.g. \textit{dennen} instead of \textit{denen};
  \item deletions, e.g. \textit{scho} instead of \textit{schon};
  \item substitutions, e.g. \textit{fur} instead of \textit{f\"ur};
\end{enumerate}
according to the type of operation which led to a particular spelling
mistake. In cases when multiple different operations were involved
simultaneously, we explicitly marked each of these operations in our
data. Statistical distribution of these subclasses is shown in Table
\ref{table:spell}.
\begin{table}
  \caption{Distribution of OOV words belonging to the class
    ``Sloppiness of users' input''\label{table:spell}}
  \begin{tabular}{p{0.4\textwidth}*{4}{>{\centering\arraybackslash}p{0.15\textwidth}}}
    \hline\noalign{\smallskip}
    \multirow{2}{*}{OOV subclass} & %
    \multicolumn{2}{c}{\texttt{hunspell}} & %
    \multicolumn{2}{c}{\texttt{TreeTagger}}\\
    & \totaloov{} & \uniqoov{} & \totaloov{} & \uniqoov{}\\
    \noalign{\smallskip} \hline
    insertions & 0.49 & 1 & 0.18 & 0.34 \\
    deletions & 8.44 & 6.38 & 6.52 & 5.27 \\
    substitutions & 2.17 & 3.37 & 1.11 & 1.2 \\\hline
    {\bfseries total} & 11.1 & 10.75 & 7.81 & 6.81\\
    \noalign{\smallskip} \hline
  \end{tabular}
\end{table}

%% As is clear from the table, deletions are by far the most common type
%% of misspellings. The reasons for that are either relatively frequent
%% omissions of characters made by users or even more often automatic
%% truncations of too long messages which are performed by Twitter
%% service itself.\footnote{As is generally known, Twitter imposes a
%%   strong restriction on the length of posted messages which can be no
%%   longer than 140 characters. Upon exceeding this length, tweets get
%%   automatically truncated to the maximal allowed length.}

Finally, the third group -- ``Stylistic specifics of text genre'' --
was subdivided into the subgroups:
\begin{enumerate}
  \item @-tokens, e.g. \textit{@ZDFonline}, \textit{@sechsdreinuller};
  \item hashtags, e.g. \textit{\#Kleinanzeigen}, \textit{\#wetter};
  \item links, e.g. \textit{http://t.co}, \textit{sueddeutsche.de};
  \item smileys, e.g. \textit{:-P}, \textit{xD};
  \item slang, e.g. \textit{OMG}, \textit{WTF} etc.
\end{enumerate}
according to the formal or lexical class which the tokens belonged
to. Additionally, we also considered as \textit{slang} spelling
variants of standard language words which could be regarded as their
colloquial equivalents. Such words included cases like, for example,
\textit{ned} instead of \textit{nicht} or \textit{grad} instead of
\textit{gerade}, and were marked both as \textit{misspelling}s and
\textit{slang} in our classification. A detailed statistics on the
subgroups of class 3 is shown in Table \ref{table:style}:
\begin{table}
  \caption{Distribution of OOV words belonging to the class
    ``Stylistic specifics of text genre''\label{table:style}}
  \begin{tabular}{p{0.4\textwidth}*{4}{>{\centering\arraybackslash}p{0.15\textwidth}}}
    \hline\noalign{\smallskip}
    \multirow{2}{*}{OOV subclass} & %
    \multicolumn{2}{c}{\texttt{hunspell}} & %
    \multicolumn{2}{c}{\texttt{TreeTagger}}\\
    & \totaloov{} & \uniqoov{} & \totaloov{} & \uniqoov{}\\
    \noalign{\smallskip} \hline
    @-tokens & 13.02 & 20.23 & 16.19 & 21.91\\
    hashtags & 7.35 & 6.18 & 13.06 & 10.59\\
    links & 2.43 & 0.4 & 4.89 & 6.07\\
    smileys & 2 & 0.73 & 6.88 & 1.2\\
    slang & 16.22 & 5.27 & 6.94 & 4.77\\\hline
    {\bfseries total} & 41.02 & 32.81 & 47.96 & 44.54\\
    \noalign{\smallskip} \hline
  \end{tabular}
\end{table}

A striking outlier of 16.22~\% for slang tokens in column 1 of the
Table is explained by the fact that the word ``RT'' which occurred
1,235 times in our texts and was by far the most frequent OOV in
analyzed data set, was recognized as out-of-vocabulary by
\texttt{hunspell} but was not deemed as such by
\texttt{TreeTagger}. Luckily, such singleton cases were rather rare
exceptions during the analyses and did not affect much the remaining
classes of OOVs, so that distributions of unknown tokens for both
tools were more or less similar or at least comparable to a certain
extent.

However, an even more important conclusion which can be drawn from the
analyzed data is the fact that for both \texttt{TreeTagger} and
\texttt{hunspell}, Twitter-specific phenomena like special tokens,
colloquial expressions or sloppily typed words accounted for more than
a half of all unknown words found during the analysis. Since different
classes of these Twitter phenomena were formed by different processes
and also have different degrees of ambiguity, we will suggest
different normalization procedures for each of them in the next
Section.

\section{Text Normalization Procedure}\label{normalization}

\subsection{Replacement of Twitter-Specific Phenomena}
According to Parker (\cite{parker}), hasthags were supposedly
introduced by Chris Messina, a renowned advocate of open-source
community, in August 2007. The ``hash godfather'', as Messina
describes himself, presumably borrowed the idea of marking relevant
topics with the ``\#''- sign from Internet Relay Chats (IRC) -- the
forebears of modern social networks -- which have been using the pound
character for marking their channel names since the early 1990s.

Luckily for us, this Messina's ``novelty'' brought along a strict
formal feature by which future hashtags could be identified. Other
types of words which also are usually considered as OOVs in Twitter
but have unambiguous formal traits are at-tokens, hyperlinks, e-mail
addresses, and smileys. The presence of such formal criteria suggests
that these classes could best be handled by rule-based methods and
namely finite-state techniques like finite-state transducers (FST)
(cf. Jurafsky and Martin, \cite{jurafsky}, pp.57-60).

For our purposes, we developed a prototypic Python system analogous to
an FST in which a set of regular expressions was associated with
corresponding actions performed on matched subgroups. If some rules'
subgroups clashed together on certain contexts, a set of special
criteria was used to determine which rule had to be preferred. These
criteria involved length of text spans matched by subgroups, length of
left and right context mathced by corresponding regular expressions,
and, finally, the order in which these expressions appeared in rule
file.

In our system, unambiguous tokens like, for example, e-mail addresses
were replaced with an artificial word ``\%Mail''. Emoticons were
substituted by tokens ``\%PosSmiley'' or ``\%NegSmiley'', depending on
the type of emotion presumably conveyed by these expressions. In cases
when an emoticon was ambiguous with regard to its polarity, it was
replaced with the special word ``\%Smiley''. Furthermore, leading
``\#{}'' characters were stripped off from the beginning of hashtags
thus leaving only the alphabetic part of these tokens.

A more complicated case turned out to be the at-tokens and
hyperlinks. For them, we had to disambiguate whether these words
played an important syntactic role in sentence or not. In the former
case, these words were replaced with artificial counterparts. In the
latter case, such words were deleted from sentence in order not to
confuse further intended parsing processing.

Precision, recall and F1 metrics for replacement of Twitter-specific
phenomena are shown in Table \ref{table:tw-phenomen-metrics}.

\begin{table}
  \caption{Precision, recall and F1 score for replacement of
    Twitter-specific phenomena\label{table:tw-phenomen-metrics}}
  \begin{tabular}{*{4}{>{\centering\arraybackslash}p{0.25\textwidth}}}
    \hline\noalign{\smallskip}
    OOV subclass & Precision & Recall & F1 score\\\hline
    attoken & 98.87 & 98.87 & 98.87\\
    link & 99.44 & 100.00 & 99.72\\
    smiley & 93.79 & 76.26 & 84.12\\
    \noalign{\smallskip} \hline
  \end{tabular}
\end{table}

All the artificial words were added to custom user dictionaries of
\texttt{TreeTagger} and \texttt{hunspell}. For tagging, we assigned NE
tags to all artificial tokens with the only exception of emoticon
replacements which got the tag INJ. Furthermore, positions and lengths
of all replacements we remembered along with the original words which
were replaced, and a restoration step to original forms was performed
for user names after tagging.

%
\subsection{Restoration of misspellings}

A closer look at unknown words which were classified as misspellings
revealed that a prevailing majority of them was considered as
colloquial spelling variants. Such spellings accounted for 70.5~\% of
detected spelling mistakes in \texttt{hunspell} data and 66.17~\% of
misspellings recognized by \texttt{TreeTagger}. Frequency distribution
of colloquial and non-colloquial misspellings for data analyzed with
either tools is shown in Figures \ref{fig:misspl-hunspell} and
\ref{fig:misspl-ttagger}.
...
%% \begin{minipage}[t]{0.45\textwidth}
%%   \begin{figure}[!ht]\label{fig:misspl-hunspell}
%%   \end{figure}
%% \end{minipage}

%% \begin{minipage}[t]{0.45\textwidth}
%%   \begin{figure}[!ht]\label{fig:misspl-ttagger}
%%   \end{figure}
%% \end{minipage}

As can be seen from the Figures, both kinds of spelling variations
have nearly Zipfian distribution. However colloquial spelling
variants, if used, tend to be used frequently whereas non-colloquial
misspellings are rather represented by sporadic singleton cases. This
could be explained by fact that colloquial spellings are usually
formed by systematic rewriting processes. These processes are supposed
to make written form of words more similar to their casual soundings
in everyday speech. According to our data, the most productive of such
processes were:
\begin{itemize}
  \item Omissions of `e' in unstressed (``schwa'') positions,
    e.g. \textit{w\"urd}, \textit{zuguckn} etc. In cases when `e' was
    part of the impersonal pronoun ``es'' and followed a verb,
    remaining `s' was usually appended to the preceding verb form;

  \item Complete omissions or replacement of final consonants with
    their voiceless equivalents, e.g. \textit{nich} instead of
    \textit{nicht} or \textit{Tach} instead of \textit{Tag};

  \item Omissions of `ei' from indefinite articles, e.g. \textit{ne}
    instead of \textit{eine} or \textit{nem} in lieu of
    \textit{einem};

  \item Multiple repetions of characters as a means of expressing
    elongation of sounds, e.g. \textit{Hilfeeee},
    \textit{s\"u\"u\"u\ss};
\end{itemize}

%
\section{Evaluation}\label{evaluation}
To evaluate our system, we used metrics

\section{Conclusion}\label{conclusion}

%% This article provided an overview of exisiting approaches to noisy
%% text normalization task. Additionally, all mentioned methods were
%% classified on the basis of two independent criteria. In section
%% \ref{error:analysis}, we performed qulitative and quantitive analyses
%% of out-of-vocabulary words in German tweets and suggested a set of
%% ad-hoc techniques for mitigating their potential negative influence on
%% natural language processing. This procedure allowed us to reduce the
%% total OOV rate by ... \% for \texttt{hunspell} and by ... \% for
%% \texttt{TreeTagger}.

%% Nevertheless, we should honestly admit that our system still has
%% potential for development and research, since it mainly addresses only
%% one of three main groups of OOV tokens. Future directions should
%% certainly include a more thorough tackling of unintentional spelling
%% mistakes and especially their most prominent types -- deletions and
%% substitutions.

%% Furthemore, a better evalution technique as well as comparison with
%% other systems are needed for our normalization procedure. On the one
%% hand, an \textit{extrinsic} evaluation should be performed (cf. Sparck
%% Jones and Galliers, \cite{sparck}) which means that we not only have
%% to show how the rate of OOV words goes down but much more how this
%% lower OOV rate affects the work of the whole NLP system. On the other
%% hand, we need to assess the quality of our procedure's work on the
%% basis of metrics used by other researchers.

%% One possible estimation criterion which was used by Aw et
%% al. (\cite{aw}), Kaufmann (\cite{kaufmann}), Beaufort et
%% al. (\cite{beaufort}), and Oliva et al. (\cite{oliva}) is the BLEU
%% score (Papineni et al., \cite{papineni}). Another possibility would be
%% to use the Word (WER) and Sentence Error Rates (SER) as suggested by
%% Kobus et al. (\cite{kobus}). However, an obvious difficulty that we
%% already encountered here is that both metrics highly rely on a
%% subjective notion of the look of a ``normalized'' message. While the
%% BLEU score could be an approriate criterion for normalization of
%% messages like ``i luv ma mather and wd do evrythin 4 her'' which in
%% fact looks like as if it were not English. But such highly distorted
%% tweets are rather atypical for German. In this regard, WER and SER
%% could be considered as more appropriate measurement criteria. But
%% these metrics once again are rather dealing with spelling mistakes and
%% would highly depend on whether one, for example, would consider the
%% hash sign in a hashtag as an error.

\subsection*{\ackname}

This work was financially supported by ... as part of collaborative
project ``...''. The authors are also thankful to ... for help with
the analysis of data.

%
% ---- Bibliography ----
%
\bibliographystyle{splncs}
\begin{thebibliography}{}
\bibitem[2006]{aw}
  Aw, A., Zhang, M., Xiao, J., Su, J.: %
  A Phrase-based Statistical Model for {SMS} Text Normalization. %
  COLING/ACL (2006) 33--40 %

\bibitem[2002]{bangalore}
  Bangalore, S., Murdock, V., Riccardi, G.: %
  Bootstrapping Bilingual Data using Consensus Translation for a %
  Multilingual Instant Messaging System. %
  COLING (2002) 33--40 %

\bibitem [2010]{beaufort}
  Beaufort, R., Roekhaut, S., Cougnon, L. A.,  Fairon, C.: %
  A Hybrid Rule/Model-Based Finite-State Framework for %
  Normalizing SMS Messages. %
  ACL (2010) 770--779 %

\bibitem[2000]{brill}
  Brill, E., Moore, R. C.: %
  An improved model for noisy channel spelling correction. %
  ACL %
  (2000) 286--293 %

\bibitem[2003]{alexander-clark}
  Clark, A.: %
  Pre-processing very noisy text. %
  In Proceedings of Workshop on Shallow Processing of Large Corpora. %
  (2003) 12--22

\bibitem[2007]{choudhury}
  Choudhury, M., Saraf, R., Jain, V., Mukherjee, A., Sarkar, S., Basu %
  A.: %
  Investigation and Modeling of the Structure of Texting Language. %
  International Journal of Document Analysis and Retrieval: Special %
  Issue on Analytics of Noisy Text. {\bfseries 10} %
  (2007) 157--174 %

\bibitem[2011]{clark-araki}
  Clark, E., Araki, K.: %
  Text Normalization in Social Media: Progress, Problems and %
  Applications for a Pre-processing System of Casual English. %
  PACLING. Procedia - Social and Behavioral Sciences {\bfseries 27} %
  (2011) 2--11

\bibitem[2009]{cook}
  Cook, P., Stevenson, S.: %
  An unsupervised model for text message normalization, %
  Proceedings of the Workshop on Computational Approaches %
  to Linguistic Creativity. %
  CALC '09. (2009) 71--78

\bibitem[2006]{dorsey}
  Dorsey, J.: %
  "just setting up my twttr". %
  \url{https://twitter.com/jack/status/20} %
  Accessed February 26, 2013. (2006)

\bibitem[2011]{han}
  Han, B., Baldwin, T.: %
  Lexical Normalization of Short Text Messages: Makn Sens %
  a \#{}twitter. %
  ACL HLT. %
  (2011) 368--378

\bibitem[2008]{jurafsky}
  Jurafsky, D., Martin, J. H.: %
  Speech and Language Processing. 2nd Edition. %
  Prentice Hall (2008) 129, 323

\bibitem [2010]{kaufmann}
  Kaufmann, M.: %
  Syntactic normalization of twitter messages. %
  The 8-th International Conference on Natural Language Processing. %
  (2010)

\bibitem [2008]{kobus}
  Kobus, C., Yvon, F., Damnati, G.: %
  Normalizing {SMS}: are Two Metaphors Better than One? %
  COLING (2008) 441--448

\bibitem[1992]{kukich}
  Kukich, K.: %
  Techniques for Automatically Correcting Words in Text. %
  ACM Computing Surveys {\bfseries 24/4} (1992) 378--439

\bibitem[2012]{guardian:sms}
  McVeigh, T.: %
  Text messaging turns 20. %
  The Observer (December 1, 2012)

\bibitem[2012]{mukherjee}
  Mukherjee, S., Malu, A., Balamurali, A. R., Bhattacharyya, P.: %
  TwiSent: A Multistage System for Analyzing Sentiment in Twitter. %
  In Proceedings of The 21st ACM Conference on Information and %
  Knowledge Management CIKM 2012, Hawai, (Oct 29 - Nov 2, 2012)


  McVeigh, T.: %
  Text messaging turns 20. %
  The Observer (December 1, 2012)

\bibitem [2013]{oliva}
  Oliva, J., Serrano, J. I., and Del Castillo, M. D., %
  and Igesias, �.: %
  A {SMS} normalization system integrating multiple %
  grammatical resources. %
  Natural Language Engineering. %
  (2013) 121--141 %

\bibitem [2002]{papineni}
  Papineni, K., Roukos, S., Ward, T., Zhu, W.-J.: %
  Bleu: a Method for Automatic Evaluation of Machine Translation. %
  ACL (2002) 311--318

\bibitem [2011]{parker}
  Parker, A.: %
  Twitter's Secret Handshake. %
  The New York Times, Page ST1, (June 10, 2011)

\bibitem [1980]{petersen}
  Petersen, L. J.: %
  Computer Programs for Detecting and Correcting Spelling Errors. %
  Communications of the ACM {\bfseries 23/ 12} (1980) 676--687

\bibitem[2012]{nyt:pope}
  Pianigiani, G., Donadio, R.: %
  Twitter Has A New User: The Pope. %
  The New York Times. Page A6. (December 4, 2012)

\bibitem[2001]{sproat}
  Sproat, R., Black, A. W., Chen, S. F., Kumar, S., %
  Ostendorf, M., Richards, Ch.: %
  Normalization of non-standard words. %
  Computer Speech \& Language, {\bfseries 15/3} (2001) 287--333

\bibitem[1994]{schmid}
  Schmid, H.: %
  Probabilistic Part-of-Speech Tagging Using Decision Trees. %
  In Proceedings of International Conference on New Methods in %
  Language Processing. (1994)

\bibitem[1996]{sparck}
  Sparck Jones, K., Galliers, J. R.: %
  Evaluating Natural Language Processing Systems. %
  An Analysis and Review. Lecture Notes in Computer %
  Science 1083, Springer. (1996)

\bibitem[2012]{terdiman}
  Terdiman, D.: %
  Report: Twitter hits half a billion tweets a day. %
  \url{http://timmurphy.org/2009/07/22/line-spacing-in-latex-documents/},%
  Accessed February 26, 2013. (2012)

\bibitem[2002]{toutanova}
  Toutanova, K., Moore, R. C.: %
  Pronunciation Modeling for Improved Spelling Correction. %
  ACL (2002) 144--151

\end{thebibliography}
%
% ---- End of Bibliography ----
%
\end{document}
